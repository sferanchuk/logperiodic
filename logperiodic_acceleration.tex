\documentclass[a4paper]{article}
\usepackage{graphicx}

\begin{document}

\begin{center}
{\Large{}}
\vskip 12pt
\end{center}

\section*{Introduction}

\noindent
{\large{\textbf{A}}}\\
\includegraphics[width=0.4\textwidth]{fig_intro_usc18.png}\\
\vskip 12pt

\noindent
{\large{\textbf{B}}}\\
\includegraphics[width=0.6\textwidth]{fig_intro_acrim3.png}\\
\vskip 12pt
Figure 1 \textit{(A) Extracts form satellite data records of photon flux deposited in Univ. of South Carolina site, radiation of Sun in 2018; (B) Radiation of Sun from 2000 to 2013 recorded by ''ACRIM3'' satellite.}

\newpage
\includegraphics[width=0.9\textwidth]{fig_sun_video.png}\\
\vskip 12pt
Figure 2 \textit{Snapshot from video record of sun flaming 15.06.2002, deposited by Swedish physicists. Digits in column - Higuchi fractal dimension for spatial axis, digits in row - approximation of fractal dimension for time axis. }

\newpage
\section*{Methods}

\subsection*{Fitting of log-periodicity}

{\large{\textbf{A}}}\\
\includegraphics[width=0.99\textwidth]{fig_methods_usc18.png}\\
\vskip 12pt

{\large{\textbf{B}}}\\
\includegraphics[width=0.99\textwidth]{fig_methods_acrim3.png}\\
\vskip 12pt
Figure 3 \textit{Illustrations of attempts to guess a presence of the log-periodic dependency (A) Uniform distribution - chart in fig. 1A (B) Periodic bursts - chart in fig 1B, two talils in separate.}
\vskip 12pt

Table 1 \textit{Supplement to figure 3 - results of fitting of log-periodicity in log-log distributions}
\vskip 12pt
\begin{tabular}{cccccccccc}
\hline
             & usc\_18        &           & acrim3-1  &           & acrim3-2  &           \\
             & method 1       & method 2  &           &           &           &           \\
\hline
plain: \\
dimension     & 0.577973  & 0.579382  & 0.691307  & 0.747027  & 0.598869  & 0.645797  \\
correlation   & -0.983925 & -0.969044 & -0.984111 & -0.97212  & -0.974277 & -0.959029 \\
\hline
fit in full: \\ 
direction     & decc.     & decc.     & accel.    & accel.    & accel.    & accel.    \\
critical time & -711      & -141      & +318      & +474      & +6        & +8        \\
dimension     & 0.686096  & 0.880372  & 0.858975  & 1.07331   & 0.881265  & 0.788433  \\
correlation   & -0.992609 & -0.996231 & -0.998645 & -0.998409 & -0.998075 & -0.979409 \\
\hline
fit in part: \\
direction     & accel.    & decc.     & decc.     & accel.    & accel.    & accel.    \\
critical time & +2        & -141      & -474      & +2        & +63       & +3        \\
dimension     & 0.677549  & 0.949008  & 0.822429  & 0.949565  & 0.599305  & 0.974975  \\
correlation   & -0.992127 & 0.996914  & -0.996929 & -0.99992  & -0.995057 & -0.999909 \\
\hline
\end{tabular}

\newpage

\subsection*{Input measurements}

\includegraphics[width=0.92\textwidth]{fig_tsi.png}\\
\vskip 12pt
Figure 4 \textit{Radiation of Sun in period from November 1978 to September 2017; in bottom - radiation at MgII frequency. }
\vskip 12pt
{\large{\textbf{A}}}\\
\includegraphics[width=0.99\textwidth]{fig_baikal_longseries.png}\\
\vskip 12pt
{\large{\textbf{B}}}\\
\includegraphics[width=0.5\textwidth]{fig_baikal_shortseries.png}\\
\vskip 12pt
Figure 5 \textit{Temperature of water in Baikal (A) series from May 2010 to October 2016; (B) series from May 2017 to September 2018. }

\section*{Results}

\includegraphics[width=0.49\textwidth]{fig_results_baikal.png}
\includegraphics[width=0.49\textwidth]{fig_results_solar.png}\\
Figure 6 \textit{Log-log dependencies, for the two types of method. A,B: temperature in Baikal, A - 2010-2016, B - 2017-2018 period, C,D: solar activity, 1978-2017, E,F - fragments of auxiliary records on solar irradiation}

\textit{\small{In A,D, three separate lines are the distributions for beginning, middle and ending parts of a period. Dashed lines in C, for a comparison - path lengths are estimated by the modified approach,  suitable to fitting of log-perodicity. In E,F, - time series are of 15.06.2002 as in fig.2, and of 01.04.2018 as in fig1B, at average and in a few randomly choosed parts. Appropriate fragments from the long-time series, shown in fig 4 and here in C,D, are added for a comparison; numerical labels are estimated least-square slopes of regression lines.}}

That is, the measure of fractal dimension is consistent for mulitiple data sources. And, the interval $s < 10$ is most sensitive (in fig 6A) to a change of time period.

\newpage

Table 2 \textit{}

\small{

\begin{tabular}{cllll}
\hline
                               & \textit{method 1}        &             & \textit{method 2}        &             \\
                               & dimension                & correlation & dimension                & correlation \\
\hline 
Solar activity,\\
\textit{total}, 1-10 t.p.\\
period 1, at a whole           & 0.611359                 & -0.98176    & -0.98176                 & -0.992606   \\
period 1, split to 10 parts    & 0.620717 $\pm$ 0.0686381 &             & 0.633088 $\pm$ 0.117143  &             \\
period 2, at a whole           & 0.582742                 & -0.980571   & 0.56841                  & -0.987462   \\
period 2, split to 10 parts    & 0.577533 $\pm$ 0.0646417 &             & 0.57153  $\pm$ 0.0621609 &             \\
period 3, at a whole           & 0.529598                 & -0.98341    & 0.940431                 & -0.961894   \\
period 3, split to 10 parts    & 0.543816 $\pm$ 0.117594  &             & 0.674605 $\pm$ 0.273296  &             \\
\textit{MnII intensity},\\
1...10 t.p., at a whole        & 0.349931                 & -0.975307   & 0.439804                 & -0.942881   \\
1...10 t.p, split to 10 parts  & 0.345247 $\pm$ 0.102797  &             & 0.35548 $\pm$ 0.195736   &             \\
1...max t.p., at a whole       & 0.531589                 & -0.960948   & 0.560595                 & -0.930574   \\
1...max t.p, split to 10 parts & 0.679449 $\pm$ 0.0622246 &             & 0.774685 $\pm$ 0.109925  &             \\
\hline
Temperature in Baikal\\
\textit{long series}; 1...10 t.p.\\
period 1, at a whole           & 0.570939                 & -0.992162   & 0.812175                 & -0.995154   \\
period 1, split to 10 parts    & 0.613501 $\pm$ 0.126926  &             & 0.868055 $\pm$ 0.177196  &             \\
period 2, at a whole           & 0.532256                 & -0.991045   & 0.739015                 & -0.994156   \\
period 2, split to 10 parts    & 0.633994 $\pm$ 0.217527  &             & 0.876311 $\pm$ 0.29755   &             \\
period 3, at a whole           & 0.303322                 & -0.99865    & 0.385959                 & -0.995293   \\
period 3, split to 10 parts    & 0.364181 $\pm$ 0.227035  &             & 0.467551 $\pm$ 0.320929  &             \\
\textit{short series}\\
1...80 t.p., at a whole        & 0.606348                 & -0.995202   & 0.662082                 & -0.988346   \\
1...80 t.p., split to 10 parts & 0.564113 $\pm$ 0.164664  &             & 0.628443 $\pm$ 0.215225  &             \\
40...80 t.p., at a whole       & 0.508056                 & -0.981986   & 0.507846                 & -0.99867   \\
40...80 t.p, split to 10 parts & 0.425593 $\pm$ 0.28323   &             & 0.444415 $\pm$ 0.282211  &             \\
\hline
\end{tabular}
}

\vskip 12pt

\includegraphics[width=0.99\textwidth]{fig_heatmaps_explained.png}\\
Figure 7 \textit{Fitting of log-periodicity. Explanation.}

\newpage

\includegraphics[width=0.99\textwidth]{fig_heatmaps_baikal.png}\\

\includegraphics[width=0.99\textwidth]{fig_heatmaps_solar.png}\\
Figure 8 \textit{Fitting of log-periodicity. Accordingly to table 2.}

That is, for data in fig 8A, a trend to contraction to up-going crisis is observed in the first row. The same trend can be seen in fig 8D for solar activity. 

\newpage

\section*{Conclusions}

The crisis on Baikal which began at around 2014 was possible to detect by slightly unusual variations of water temperature in two to four years before (fig 9 A). 

If a covid-19 is not an ordinary virus pandemy, to what extent is this extraaordiunarity expanded? The exclusive search gave similarity of ''projection'' to a model of log-periodic acceleration between variations of temperature in Baikal before crisis, and variations of intensity of solar irradiation in past 30 years (fig. 9B), this can point out to expectations of universe-scale crisis in a near future.

\vskip 12pt

\includegraphics[width=0.7\textwidth]{fig_conclusions.png}\\
Figure 9 \textit{Concluding drawing.}

\section*{References}

\begin{enumerate}

\item Nottale, L., Scale relativity and fractal space-time: theory and applications, \textit{arxiv.org}, 2008

\item Feranchuk, S., Belkova, N., et al. \textit{Limnology and Freshwater Biology}, 2018,

\end{enumerate}

\newpage
\section*{Appendix A}


\texttt{\small{ cat usc\_18.txt | awk -v i=0 -v b1=2458119.5 -v b13=7 '\{ if ( i == 100 \&\& substr( \$13, 1, 1 ) != "0" ) \{ s = s "," 10 * ( \$1 - b1 ) "," 500 * (substr(\$13,1,7) - b13 ); i = 0; \}; i = i+1; \} END \{ print substr( s, 2 ) \}' | ./fractal\_dimension -d\_xy }}

%grep ^20 ./daya2sddeg\_ts4_Nov\_2013\_hdr.txt | awk -v b1=2000 -v b3=1360 '{ s=s "," 100 *( $1 - b1 ) "," 100 * ( $3 - b3 ); } END { print substr(s,2,length(s)); }' | ./fractal\_dimension -d\_xy 

%echo "$s_beg" >../figure_sources/fig_baikal_shortseries.html;cat ../data_files/baikal_shortseries.txt | while read line; do a1=${line:0:10};a2=${line:11:5};n1=`echo "($a1-$d_beg)*0.00001" | bc`;n2=`echo "$a2*10" | bc`; s="+\" $n1,$n2\""; echo "$s" >>../figure_sources/fig_baikal_shortseries.html; done; echo "$s_end" >>../figure_sources/fig_baikal_shortseries.html;

%head -n 525252 ../data_files/baikal_longseries.txt |   awk -F ',' '{ print ($1-1262293200)*0.000001 "," $2 }'
%head -n 1222863 ../data_files/baikal_longseries.txt | tail -n $((1222863-718437))  |  awk -F ',' '{ print ($1-1367352120)*0.000001 "," $2 }'
%tail -n $((1575756-1222863)) ../data_files/baikal_longseries.txt |  awk -F ',' '{ print ($1-1430427720)*0.000001 "," $2 }'
%cat ../data_files/baikal_shortseries.txt |   awk -F ',' '{ print ($1-1496172600)*0.000001 "," $2 }'

%for cmd in './fractal_dimension -lh -v -max 10 -linincr' './fractal_dimension -lh -v -max 1000' './fractal_dimension -lh -v'; do echo "["; cat ../data_files/lin_tsi.txt | bash -c "$cmd" | head -n 38 | tail -n 36; echo "],"; done;

%head -n 525252 ../data_files/baikal_longseries.txt |  awk -F "," '{ print (1367352120-$1)*0.000001*(log(2+(1367352120-$1)/(86400*365))/(log(4)-log(2))) "," $2 }' | ./fractal_dimension -hm -v -xy
%cat ../data_files/lin_mn.txt | awk -F "," '{ for ( i=1; i <= NF; i+= 2 ) { print (365*(36-$i/365)*log(36-$i/365)/log(10)) "," $(i+1); } }' | ./fractal_dimension -hm -v -xy

\end{document}
