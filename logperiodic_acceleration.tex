\documentclass[a4paper]{article}
\usepackage{graphicx}

\begin{document}

\begin{center}
{\Large{}}
\vskip 12pt
\end{center}

\section*{Introduction}

{\large{\textbf{A}}}\\
\includegraphics[width=0.4\textwidth]{fig_intro_usc18.png}\\
\vskip 12pt
{\large{\textbf{B}}}\\
\includegraphics[width=0.6\textwidth]{fig_intro_acrim3.png}\\
\vskip 12pt
Figure 1 \textit{(A) Extracts form satellite data records of photon flux deposited in Univ. of South Carolina site, radiation of Sun in 2018; (B) Radiation of Sun from 2000 to 2013 recorded by ''ACRIM3'' satellite.}

\newpage
\includegraphics[width=0.9\textwidth]{fig_sun_video.png}\\
\vskip 12pt
Figure 2 \textit{Snapshot from video record of sun flaming 15.06.2002, deposited by Swedish physicists. Digits in column - Higuchi fractal dimension for spatial axis, digits in row - approximation of fractal dimension for time axis. }

\newpage
\section*{Methods}

{\large{\textbf{A}}}\\
\includegraphics[width=0.99\textwidth]{fig_methods_usc18.png}\\
\vskip 12pt
{\large{\textbf{B}}}\\
\includegraphics[width=0.99\textwidth]{fig_methods_acrim3.png}\\
\vskip 12pt
Figure 3 \textit{Illustrations of attempts to guess a presence of the log-periodic dependency (A) Uniform distribution - chart in fig. 1A (B) Periodic bursts - chart in fig 1B, two talils in separate.}
\vskip 12pt

Table 1 \textit{Supplement to figure 3 - results of fitting of log-periodicity in log-log distributions}
\vskip 12pt
\begin{tabular}{cccccccccc}
\hline
             & usc\_18        &           & acrim3-1  &           & acrim3-2  &           \\
             & method 1       & method 2  &           &           &           &           \\
\hline
plain: \\
dimension     & 0.577973  & 0.579382  & 0.691307  & 0.747027  & 0.598869  & 0.645797  \\
correlation   & -0.983925 & -0.969044 & -0.984111 & -0.97212  & -0.974277 & -0.959029 \\
\hline
fit in full: \\ 
direction     & decc.     & decc.     & accel.    & accel.    & accel.    & accel.    \\
critical time & -711      & -141      & +318      & +474      & +6        & +8        \\
dimension     & 0.686096  & 0.880372  & 0.858975  & 1.07331   & 0.881265  & 0.788433  \\
correlation   & -0.992609 & -0.996231 & -0.998645 & -0.998409 & -0.998075 & -0.979409 \\
\hline
fit in part: \\
direction     & accel.    & decc.     & decc.     & accel.    & accel.    & accel.    \\
critical time & +2        & -141      & -474      & +2        & +63       & +3        \\
dimension     & 0.677549  & 0.949008  & 0.822429  & 0.949565  & 0.599305  & 0.974975  \\
correlation   & -0.992127 & 0.996914  & -0.996929 & -0.99992  & -0.995057 & -0.999909 \\
\hline
\end{tabular}

\newpage
\section*{Results}

\includegraphics[width=0.92\textwidth]{fig_tsi.png}\\
\vskip 12pt
Figure 4 \textit{Radiation of Sun in period from November 1978 to September 2017; in bottom - radiation at MgII frequency. }

\newpage
{\large{\textbf{A}}}\\
\includegraphics[width=0.99\textwidth]{fig_baikal_longseries.png}\\
\vskip 12pt
{\large{\textbf{B}}}\\
\includegraphics[width=0.5\textwidth]{fig_baikal_shortseries.png}\\
\vskip 12pt
Figure 5 \textit{Temperature of water in Baikal (A) series from May 2010 to October 2016; (B) series from May 2017 to September 2018. }


\vskip 12pt
\includegraphics[width=0.45\textwidth]{fig_results_baikal.png}
\includegraphics[width=0.45\textwidth]{fig_results_solar.png}\\
Figure 6 \textit{Log-log dependencies, for the two types of method. A,B: temperature fn Baikal, A - 2010-2016, B - 2017-2018 period, C,D: solar activity, 1978-2017, E,F - fragments of auxiliary records on solar irradiation}

\textit{\small{In A,D, three separate lines are the distributions for beginning, middle and ending parts of a period. Dashed lines in C, for a comparison - path lengths are estimated by the modified approach,  suitable to fitting of log-perodicity. In E,F, - time series are of 15.06.2002 as in fig.2, and of 01.04.2018 as in fig1B, at average and in a few randomly choosed parts. Appropriate fragments from the long-time series, shown in fig 4 and here in C,D, are added for a comparison; numerical labels are estimated least-square slopes of regression lines.}}

\section*{References}

\begin{enumerate}

\item Nottale, L., Scale relativity and fractal space-time: theory and applications, \textit{arxiv.org}, 2008

\item Feranchuk, S., Belkova, N., et al. \textit{Limnology and Freshwater Biology}, 2018,

\end{enumerate}

\newpage
\section*{Appendix A}


\texttt{\small{ cat usc\_18.txt | awk -v i=0 -v b1=2458119.5 -v b13=7 '\{ if ( i == 100 \&\& substr( \$13, 1, 1 ) != "0" ) \{ s = s "," 10 * ( \$1 - b1 ) "," 500 * (substr(\$13,1,7) - b13 ); i = 0; \}; i = i+1; \} END \{ print substr( s, 2 ) \}' | ./fractal\_dimension -d\_xy }}

%grep ^20 ./daya2sddeg\_ts4_Nov\_2013\_hdr.txt | awk -v b1=2000 -v b3=1360 '{ s=s "," 100 *( $1 - b1 ) "," 100 * ( $3 - b3 ); } END { print substr(s,2,length(s)); }' | ./fractal\_dimension -d\_xy 

%echo "$s_beg" >../figure_sources/fig_baikal_shortseries.html;cat ../data_files/baikal_shortseries.txt | while read line; do a1=${line:0:10};a2=${line:11:5};n1=`echo "($a1-$d_beg)*0.00001" | bc`;n2=`echo "$a2*10" | bc`; s="+\" $n1,$n2\""; echo "$s" >>../figure_sources/fig_baikal_shortseries.html; done; echo "$s_end" >>../figure_sources/fig_baikal_shortseries.html;

\end{document}
